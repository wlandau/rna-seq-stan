\documentclass[handout]{beamer}
\usetheme{Marburg}
\useoutertheme{infolines}
\newcommand{\answers}{1}

\usepackage{amsmath}
\usepackage{caption}
\usepackage{color}
\usepackage{enumerate}
\usepackage{listings}
\usepackage{hyperref}
\usepackage{mathrsfs}
\usepackage{setspace}
\usepackage{tikz}
\usepackage{tkz-graph}
\usepackage{url}

\providecommand{\all}{\ \forall \ }
\providecommand{\bs}{\backslash}
\providecommand{\e}{\varepsilon}
\providecommand{\E}{\ \exists \ }
\providecommand{\lm}[2]{\lim_{#1 \rightarrow #2}}
\providecommand{\m}[1]{\mathbb{#1}}
\providecommand{\mc}[1]{\mathcal{#1}}
\providecommand{\nv}{{}^{-1}}
\providecommand{\ov}[1]{\overline{#1}}
\providecommand{\p}{\newpage}
\providecommand{\q}{$\quad$ \newline}
\providecommand{\rt}{\rightarrow}
\providecommand{\Rt}{\Rightarrow}
\providecommand{\vc}[1]{\boldsymbol{#1}}
\providecommand{\wh}[1]{\widehat{#1}}

\hypersetup{colorlinks,linkcolor=,urlcolor=blue}
\numberwithin{equation}{section}

\definecolor{dkgreen}{rgb}{0,0.6,0}
\definecolor{gray}{rgb}{0.5,0.5,0.5}
\definecolor{mauve}{rgb}{0.58,0,0.82}

\lstset{ 
  language=C,                % the language of the code
  basicstyle= \footnotesize,           % the size of the fonts that are used for the code
  numbers=left,
  numberfirstline=true,
  numbersep=5pt,                  % how far the line-numbers are from the code
  backgroundcolor=\color{white},      % choose the background color. You must add \usepackage{color}
  showspaces=false,               % show spaces adding particular underscores
  showstringspaces=false,         % underline spaces within strings
  showtabs=false,                 % show tabs within strings adding particular underscores
  frame=lrb,                   % adds a frame around the code
  rulecolor=\color{black},        % if not set, the frame-color may be changed on line-breaks within not-black text 
  tabsize=2,                      % sets default tabsize to 2 spaces
  captionpos=t,                   % sets the caption-position 
  breaklines=true,                % sets automatic line breaking
  breakatwhitespace=false,        % sets if automatic breaks should only happen at whitespace
  %title=\lstname,                   % show the filename of files included with \lstinputlisting;
  keywordstyle=\color{blue},          % keyword style
  commentstyle=\color{gray},       % comment style
  stringstyle=\color{dkgreen},         % string literal style
  escapeinside={\%*}{*)},            % if you want to add LaTeX within your code
  morekeywords={*, ...},               % if you want to add more keywords to the set
  xleftmargin=0.053in, % left horizontal offset of caption box
  xrightmargin=-.03in % right horizontal offset of caption box
}

\DeclareCaptionFont{white}{\color{white}}
\DeclareCaptionFormat{listing}{\parbox{\textwidth}{\colorbox{gray}{\parbox{\textwidth}{#1#2#3}}}}
\captionsetup[lstlisting]{format = listing, labelfont = white, textfont = white}
 %For caption-free listings, comment out the 3 lines above
 \lstset{frame = single}

\title{The heterosis problem: a comparison of Eric's method with {\tt edgeR}, {\tt baySeq}, and {\tt ShrinkBayes}}
\author{Will Landau, Eric Mittman}
\date{\today}
\institute{Iowa State University}

\begin{document}

\begin{frame}
\titlepage
\end{frame}

 \AtBeginSection[]
{
   \begin{frame}
       \frametitle{Outline}
       \tableofcontents[currentsection]
   \end{frame}
}

\section{Simulated data}

\begin{frame}
\frametitle{Simulate heterosis data with known heterosis genes}
\begin{center}
\includegraphics[scale=.28]{data}
\end{center}
\end{frame}

\begin{frame}
\frametitle{Apply {\tt edgeR} to real data to get simulation parameters}
\begin{center}
\includegraphics[scale=.25]{parms}
\end{center}
\end{frame}


\begin{frame}
\frametitle{Truth: which genes have heterosis?}
\begin{center}
\includegraphics[scale=.28]{truth}
\end{center}
\end{frame}

\begin{frame}
\frametitle{iid negative binomial counts (parent 1)}
\begin{center}
\includegraphics[scale=.28]{dists}
\end{center}
\end{frame}

\begin{frame}
\frametitle{Remove low-count rows to get 25000 features}
\begin{center}
\includegraphics[scale=.28]{data}
\end{center}
\end{frame}

\begin{frame}
\frametitle{Simulation workflow}

\begin{itemize}
\item Simulate 30 datasets as above:
\begin{itemize}
\pause \item 10 datasets with 4 samples (libraries, columns, etc.) per group
\item 10 with 8 per group
\item 10 with 16 per group
\end{itemize}
\pause \item For each simulated dataset, test for heterosis with
\begin{itemize}
\item empirical Bayes with STAN (Eric's method)
\item {\tt edgeR} 
\item {\tt baySeq}
\item {\tt ShrinkBayes}
\end{itemize}
\item Compare methods with ROC curves
\end{itemize}
\end{frame}

\begin{frame}
\frametitle{{\tt edgeR}}

\begin{itemize}
\item Apply a loglinear model to get estimates of main effects $\mu_{f, t}$
\begin{itemize}
\item Feature $f = 1, \ldots, 25000$
\item Treatment group 1 (parent), 2 (parent), 3 (hybrid)
\end{itemize}
\pause \item For each $f$, calculate a p-value
\begin{itemize}
\item p-values is 1 if $\mu_{f, 3}$ is between $\mu_{f, 1}$ and $\mu_{f, 2}$
\end{itemize}

\end{itemize}

\end{frame}


\end{document}